\chapter*{Úvod}
\addcontentsline{toc}{chapter}{Úvod}
V dnešní době je internet synonymem pro používání počítače, tabletu, smartphone a jiných zařízení. S tím rozhodně souvisí otázka bezpečnosti uživatelských dat. Většina uživatelů bohužel využívá všude stejné heslo, proto obezřetnému útočníkovi stačí získat login a heslo z jedné databáze a zkusit to \\i jinde. K této situace došlo nedávno při napadení porno stránek (\textit{pron.com}) skupinou \uv{LulzSec}\footnote{Lulz Security - tato skupina stála i za útokem na Sony Pictures v roce 2011, kde právě díky SQL Injection odcizila velké množství dat.}, která následně získaná data zveřejnila\cite{examiner}. Hesla nebyla v databázích nijak \uv{hashována}\footnote{Hash - algoritmus pro převedení vstupních dat do unikátního otisku, tato funkce \uv{by měla být jednosměrná!}}, byla uložena v podobě otevřeného textu, a proto nebyl problém vyzkoušet se přihlásit do emailových schránek, popřípadě dalších jiných webových služeb, které tito uživatelé využívali. Nikdy přesně nevíme, komu vlastně data svěřujeme a jaké bezpečnostní opatření je dotyčnou firmou či osobou zajištěno! Data jsou uchovávána v mnoha databázových systémech a jednou z hlavních otázek je také bezpečnost těchto dat. Možností útoků na webové aplikace, webové stránky nebo přímo servery je mnoho. Významnější budou popsány v následující kapitole.

Cílem této práce je seznámit čtenáře s bezpečnostními riziky webových aplikací, metodami prevence rizik a dále podrobnější seznámení s problémem SQL injection. Cílem praktické části bude vytvoření nástroje pro penetrační testy určeného k detekci těchto bezpečnostních rizik.

Celá tato práce je především zaměřena na detekci problému SQL injection. Aktuálně je podle serveru \textit{http://techworld.com} za zhruba 97\% úniků dat zneužití právě této chyby. 