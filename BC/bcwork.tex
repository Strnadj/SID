\documentclass[12pt, a4paper]{report}
\usepackage{lmodern}
\usepackage[T1]{fontenc}
\usepackage[czech]{babel}
\usepackage[utf8]{inputenc}
\usepackage{graphicx}
\usepackage{hyperref}
\usepackage{color}
\usepackage{xcolor}
\usepackage{fancyhdr}
\usepackage{listings}
\lstloadlanguages{Ruby}
\lstset{language=Ruby,
	basicstyle=\ttfamily\color{black},
	commentstyle = \ttfamily\color{red},
	keywordstyle=\ttfamily\color{blue},
	stringstyle=\color{orange},
	numberstyle=\footnotesize,
	basicstyle=\ttfamily\footnotesize,
	breaklines=true,
	showspaces=false             
	language=ruby,
        basicstyle=\scriptsize
}
\lstdefinelanguage{JavaScript}{
  keywords={typeof, new, true, false, catch, function, return, null, catch, switch, var, if, in, while, do, else, case, break},
  keywordstyle=\color{blue}\bfseries,
  ndkeywords={class, export, boolean, throw, implements, import, this},
  ndkeywordstyle=\color{darkgray}\bfseries,
  identifierstyle=\color{black},
  sensitive=false,
  comment=[l]{//},
  morecomment=[s]{/*}{*/},
  commentstyle=\color{purple}\ttfamily,
  stringstyle=\color{red}\ttfamily,
  morestring=[b]',
  morestring=[b]"
}
\usepackage{caption}
\DeclareCaptionFont{white}{\color{white}}
\DeclareCaptionFormat{listing}{\colorbox{gray}{\parbox{\textwidth}{#1#2#3}}}
\captionsetup[lstlisting]{format=listing,labelfont=white,textfont=white}

\usepackage{titlesec}
\titlespacing*{\chapter}{0pt}{-10pt}{20pt}
\titleformat{\chapter}[display]{\normalfont\huge\bfseries}{\chaptertitlename\ \thechapter}{20pt}{\Huge}


\hypersetup{
	linkcolor=black,
	urlcolor=blue,
	colorlinks=false
}

%%%%%%%%%%%%%%%%%%%%%%%%%%%%%%%%%%
% 			DEFINICE SEZNAMU 			%
%%%%%%%%%%%%%%%%%%%%%%%%%%%%%%%%%%
%% Vysazeni seznamu obrazku
\newcommand{\seznamobrazku}%
  {{% prostredi kvuli \pagestyle
    \pagestyle{empty}
    \if@twoside\cleardoublepage\else\clearpage\fi
    \phantomsection 
    \addcontentsline{toc}{chapter}{Seznam obrázků}
    \listoffigures
    \newpage
	
    \if@twoside
      \pagestyle{empty}
      \cleardoublepage
    \fi
  }}%


%% Vysazeni seznamu tabulek
\newcommand{\seznamtabulek}%
  {{% prostredi kvuli \pagestyle
    \pagestyle{empty}
    \if@twoside\cleardoublepage\else\clearpage\fi
    \phantomsection 
    \addcontentsline{toc}{chapter}{Seznam tabulek}
    \listoftables
    \newpage

    \if@twoside
      \pagestyle{empty}
      \cleardoublepage
    \fi
  }}%

\renewcommand\lstlistingname{Kód}
\renewcommand\lstlistlistingname{Seznam algoritmů}

%% Vysazeni seznamu algoritmiu
\newcommand{\seznamalgoritmu}%
  {{% prostredi kvuli \pagestyle
    \pagestyle{empty}
    \if@twoside\cleardoublepage\else\clearpage\fi
    \phantomsection 
    \addcontentsline{toc}{chapter}{Seznam algoritmů}
    \lstlistoflistings
    \newpage

    \if@twoside
      \pagestyle{empty}
      \cleardoublepage
    \fi
  }}%


%****************************************************************************************************%
% definice zahlavi pomoci stylu FANCYHDR.STY (nevztahuje se na stranku s "\chapter")
\fancyhf{}                            % vymazani vsech zahlavi a zapati (left/center/right header and footer)
\lhead{\textsl{\rightmark}}           % leve zahlavi -- neprave kurziva (slanted) -- nazev a cislo kapitoly
\rhead{\thepage}                      % prave zahlavi -- cislo stranky
\pagestyle{fancy}                     % format stranky bude podle FANCYHDR.STY
\renewcommand{\headrulewidth}{0.4pt}  % tloustka cary, ktera oddeluje zahlavi od textu
\renewcommand{\footrulewidth}{0pt}    % tloustka cary, ktera oddeluje zapati od textu
\renewcommand{\sectionmark}[1]{\markright{\thesection\ #1}}  % Zadne tecky za cisly sekci v zahlavi

% predefinovane zahlavi a zapati pro styl "plain" pomoci stylu FANCYHDR.STY
% (vztahuje se i na stranku s "\chapter")
\fancypagestyle{plain}{
  \fancyhf{}                 % vymazani vsech zahlavi a zapati (left/center/right header and footer)
  \renewcommand{\headrulewidth}{0.4pt} % tloustka cary, ktera oddeluje zahlavi od textu
  \renewcommand{\footrulewidth}{0pt}  % tloustka cary, ktera oddeluje zapati od textu
  \rhead{\thepage}           % prave zahlavi -- cislo stranky
}


%****************************************************************************************************%

\begin{document}

% Author info
\author{Jan Strnádek}
\date{10.10.2012}
\title{KIV/PRJ5\\SID\\\small{SQL Injection Attack Detector\\Tento program je určen pouze k školním a testovacím účelům! Je zakázáno ho využívat k nelegální činnosti a autor ani ZČU nenese jakoukoliv zodpovědnost škodám způsobeným využitím softwaru a ani jeho součástí pro nelegální účely}}

% Title page
\input includes/title

% TOC
\tableofcontents
\newpage

% Prohlaseni
\input includes/prohlaseni
\newpage

% Abstrakt
\input includes/abstract

% Content
\input includes/introduction

% Overview
\input includes/overview

% SQL
\input includes/sqli

% Pentest
\input includes/pentest

% Compare and use
\input includes/compare_and_use

% End
\input includes/end

%%%%% LIST - bibigliography, sources, tables, images
\phantomsection
\begin{thebibliography}{9}
\addcontentsline{toc}{chapter}{Použitá literatura a zdroje}
\bibitem{owasp}{\em OWASP community}
	{\bf OWASP Wiki} [on-line] \\
	\texttt{https://www.owasp.org/}, 2013

\bibitem{soom.cz}{\em Komunitní server}
	{\bf Soom.cz - Pokročilé techniky XSS} [on-line]\\
	\texttt{http://www.soom.cz/index.php?name=articles/show\&aid=485\&\\title=Pokrocile-techniky-XSS}, 2008

\bibitem{cgisecurity}{\em Robert Auger}
	{\bf Cgisecurity.com} [on-line]\\
	\texttt{http://www.cgisecurity.com/csrf-faq.html}, 2010

\bibitem{owasppt}{\em OWASP comunity} {\bf OWASP Wiki - Path traversal}\\
	\texttt{https://www.owasp.org/index.php/Path$\_$Traversal}, 2013 [on-line]

\bibitem{ruby}{\em Ruby comunnity}
	{\bf Ruby doc} [on-line]\\
	\texttt{http://www.ruby-lang.org/en/documentation/}, 2013

\bibitem{examiner}{\em Michael Santo}
	{\bf Examiner.com} [on-line]\\
	\texttt{http://www.examiner.com/article/lulzsec-posts-email-addres\\ses-passwords-for-26-000-porn-site-users}

\bibitem{LearnRubyHardWay}{\em Zed A. Shaw and Rob Sobers} {\bf Learn Ruby The Hard Way rev.2}\\
	\texttt{http://programming-motherfucker.com/}, 2012 E-book

\bibitem{HackingBezTajemstvi}{\em Joel Scambray, Stuart McClure, George Kurtz}
               {\bf Hacking bez tajemství} \\
           	Computer Press, 2010

\bibitem{php}{\em The PHP Group}
	{\bf PHP - Documentation} [on-line]\\
	\texttt{http://php.net/manual/en/}, 2013

\bibitem{exploitdb}{\em Offensive Security}
	{\bf Exploit-Db} [on-line]\\
	\texttt{http://www.exploit-db.com/webapps/}, 2013

\bibitem{fos}{\em Neil Daswani, Christoph Kern and Anita Kesavan}
	{\bf Foundations of Security} [on-line]\\
	Apress, 2007

\bibitem{guardian}{\em Charles Arthur, Dan Sabbagh and Sandra Laville
} {\bf The Guardian}\\
	\texttt{http://www.guardian.co.uk/technology/2012/mar/06/lulz\\sec-sabu-working-for-us-fbi}, 2012 [on-line]
\end{thebibliography}

% Vysázení seznamu obrázků
\seznamobrazku

% Vysázení seznamu tabulek
\seznamtabulek

% Seznam algoritmu
%\seznamalgoritmu
\seznamalgoritmu

\end{document}